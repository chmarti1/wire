\section{Integration}\label{sec:integral}

We will calculate values of $J_D$ at nodes with uniform spacing, $d$ in both $x$ and $y$ directions.  Though the choice of uniform spacing is not mandatory for this approach, it simplifies matters while sacrificing little generality.  Our problem may be said to contain $N_n$ nodes forming $N_e$ $\delta\times \delta$ square elements when the array of nodes contains $N_x$ $x$-values and $N_y$ $y$-values, and
\begin{align}
N_n &= N_x N_y\\
N_e &= (N_x -1)(N_y -1).
\end{align}
Nodes may be indexed either sequentially (using index $n$) from left-to-right and then bottom-to-top, or by $x$- and $y$-indices (using indices $i$ and $j$).  The equivalent indices may be found
\begin{align}
n &= jN_x + i\\
j &= \mathrm{floor}(n/N_x)\\
i &= n - jN_x
\end{align}
when $n \in [0, N_n-1]$, $i \in [0, N_x-1]$, and $j \in [0, N_y-1]$.  

We will take $\X$ to be a vector of all node values for $J_D$, which can be indexed either in sequential order, $X_n$, or by $x$- and $y$-indices, $X_{i,j}$.  Elements are numbered to match the index of the node in the bottom-left corner, so that $X_n$ is the bottom-left value of element $(n)$.  However, it is important to note that the top- and right-most elements cannot be evaluated since they will be missing nodes.

When our choices for all $\X$ are optimal,
\begin{align}
0 &= \frac{\partial E^2}{\partial X_n} \nonumber\\
& = \sum_k 2 e_k\frac{\partial e_k}{X_n} \nonumber\\
&= \sum_k \left(-I_k + \pi D_w \int_0^R J_D(\p_k(r)) \mathrm{d}r \right)\int_0^R \frac{\partial J_D(\p_k(\zeta))}{\partial X_n}\mathrm{d}\zeta\label{eqn:dEdJ}
\end{align}
for all $n$.

There are two integrals that need to be evaluated in order to satisfy Equation \ref{eqn:dEdJ}:
\begin{align}
\int_0^R J_D(\p) \mathrm{d}r & & \int_0^R \frac{\partial J_D}{\partial X_n} \mathrm{d}r\nonumber
\end{align}
For a given measurement $k$, the wire's trajectory through the domain will only pass through a minority of the elements, so if we can develop a scheme for calculating these quantities in a single element, we may sum the results to obtain the total integral.

\subsection{Interpolation}\label{sec:interpolate}

Recall that we have defined the value $X_n$ at nodes, which form the corners of elements.  Rather than sequentially, we might alternately index the nodes by their location in the $x,y$ grid, $X_{i,j}$.  Each element is a finite domain in $x,y$ space over which we may estimate the continuous function, $J_D(\p)$, by linear interpolation of the node values.  
\begin{align}
J_D(\p) &= \nonumber\\
 &X_{i,j}\,\phi_{00} (\p) + X_{i+1,j}\,\phi_{10} (\p) + \dots\nonumber\\
 &X_{i,j+1}\,\phi_{01} (\p) + X_{i+1,j+1}\,\phi_{11} (\p)
\end{align}
when the $\phi$ functions are interpolation functions.  

A formulation of the interpolation functions is facilitated by using a scaled coordinate system, $\hat{x} = (x-x_i) / \delta$, $\hat{y} = (y-y_j)/\delta$, or
\begin{align}
[\hat{x}, \hat{y}]^T = \hat{\p} = \frac{\p - \p_n}{\delta},
\end{align}
This constitutes a coordinate system with its origin at the bottom-left most node in the element (node $n$ or equivalently node $i,j$) extending to $\hat{x}=1$ at its right-most, and $\hat{y} = 1$ at its top most extent.  These interpolation functions are \emph{only} valid in that range.

Using the element-scaled coordinate system, the interpolation functions are
\begin{subequations}
\begin{align}
\phi_{00} &= (1-\hat{x})(1-\hat{y})\\
\phi_{10} &= \hat{x}(1-\hat{y})\\
\phi_{01} &= (1-\hat{x})\hat{y}\\
\phi_{11} &= \hat{x}\hat{y}
\end{align}
\end{subequations}
and may be more compactly written $\phi(\hat{\p})$.  Note that each $\phi$ is unity when $\hat{\p}$ is at its respective node, but declines to zero at all other nodes.

\subsection{Line Segments}\label{sec:segments}

Constructing the wire's path in space and the bounds on an element's regime in space is accomplished by defining line segments.  In the case of elements, the four segments defining their boundary are simply defined by the segments connecting the nodes.  In the case of the wire, the wire may be imagined to extend a radius, $R$, from the center of rotation at some angle relative to the $x$-axis.

It is convenient to define a line segment by a starting point, $\p_0$, and a direction, $\Delta \p$.  Therefore, the line segment is defined as
\begin{align}
\p(s) = \p_0 + s \Delta \p \hspace{2em} \forall \ s\in[0,1].
\end{align}
Negative values of $s$ and values greater than 1 represented points projected beyond the bounds of the line segment.  The value of $s$ is related to the distance along the segment, $r$, by
\begin{align}
r = s \| \Delta \p \|.
\end{align}

To calculate the location of an intersection between two line segments, $a$ and $b$,
\begin{align}
\p_a(s_a) &= \p_{0,a} + s_a \Delta \p_a\nonumber\\
\p_b(s_b) &= \p_{0,b} + s_b \Delta \p_b\nonumber,
\end{align}
we need only set the points $\p_a$ and $\p_b$ equal and solve for $s_a$ and $s_b$.  The problem is equivalent to
\begin{align}
\p_{0,b} - \p_{0,a} &= \left[\Delta \p_a,\ -\Delta \p_b\right] \cdot \left[s_a,\ s_b\right]^T
\end{align}
or, in terms of the individual $x$ and $y$ components,
\begin{align}
\left[\begin{array}{c}
x_{0,b} - x_{0,a}\\
y_{0,b} - y_{0,b}
\end{array}\right] &= 
\left[\begin{array}{cc}
\Delta x_a & -\Delta x_b\\
\Delta y_a & -\Delta y_b
\end{array}\right]\left[\begin{array}{c}
s_a \\
s_b
\end{array}\right]
\end{align}

There are five cases that can occur when searching for segment intersections:
\begin{enumerate}
\item If the determinant of the matrix, $\Delta x_b \Delta y_a - \Delta x_a \Delta y_b$ is zero, then the segments are parallel, and there is no solution,
\item If $0 \le s_a \le 1$ and $0 \le s_b \le 1$ then the segments intersect,
\item If $0 \le s_a \le 1$ but not $s_b$, then a projection of segment $b$ intersects segment $a$,
\item If $0 \le s_b \le 1$ but not $s_a$, then a projection of segment $a$ intersects segment $b$,
\item If neither $s_a$ nor $s_b$ is between 0 and 1, then only the projections of the segments intersect.
\end{enumerate}

For line segments expressed within an element's dimensionless coordinate system,
\begin{align}
\hat{\p}(s) &= \frac{\p(s) - \p_n}{\delta} = \hat{\p}_0 + s\Delta \hat{\p}\\
\hat{\p}_0 &= \frac{\p_0 - \p_n}{\delta}\nonumber\\
\Delta \hat{\p} &= \frac{\Delta \p}{\delta}\nonumber.
\end{align}
This alternate formulation for a line segment is simply re-scaled by the elements size and offset by its bottom-left element's coordinates.

\subsection{Integration within an element}\label{sec:element}

We have established that the integrals from Equation \ref{eqn:dEdJ} will be evaluated over each element, we have established a description for how $J_D$ should be evaluated as a field variable within the element, and we have established how we will formulate paths through space.

For a given wire location, elements may fall into three categories:
\begin{enumerate}
\item Elements through which the wire does not pass,
\item Elements with the wire passing through two faces,
\item An element with the wire passing through only one face.
\end{enumerate}
The first will constitute the vast majority of elements for any given wire location and do not contribute to the integral.  The only element that will be of the last type will contain the wire tip.  The last two types will need to be evaluated using the approach here.

The integration of $J_D$ within the element will contain contributions from the four nodes that form its limits,
\begin{align}
\int_{(n)} J_D \mathrm{d} r &= X_{i,j} \int_{(n)} \phi_{00} \mathrm{d} r + X_{i+1,j} \int_{(n)} \phi_{10} \mathrm{d} r + \ldots \\
&\ X_{i,j+1} \int_{(n)} \phi_{01} \mathrm{d} r + X_{i+1,j+1} \int_{(n)} \phi_{11} \mathrm{d} r \label{eqn:intJD:element}
\end{align}

Note that we have been quite deliberately vague with regard to the bounds on the integrals over $r$.  The integrals should be constructed to be over the section of the wire line segment that lies inside the element.  We could carefully construct a notation for values of $r$ that lie in this range, but we shall opt for the functional (albeit lazier) approach of merely using $(n)$ for this purpose.  

For each of the integrals presumed to be limited to element $n$, we may translate the bounds to an integral on $s$,
\begin{align}
\int_{(n)} \phi \mathrm{d}r = \|\Delta\p\| \int_0^1 \phi(\hat{p}(s)) \mathrm{d} s.
\end{align}
So, the four interpolation function integrals are
\begin{subequations}
\begin{align}
\Phi_{00} &= \int_0^1 \phi_{00}(s) \mathrm{d}s = \int_0^1 (1-\hat{x}_0 - s\Delta\hat{x})(1-\hat{y}_0-s\Delta\hat{y})\mathrm{d}s\nonumber\\
 &= \frac{\Delta \hat{x} \Delta \hat{y}}{3} - \frac{\Delta \hat{x} (1-\hat{y}_0)}{2} - \frac{\Delta \hat{y} (1-\hat{x}_0)}{2} + (1-\hat{x}_0)(1-\hat{y}_0)\\
\Phi_{10} &= \int_0^1 \phi_{10}(s) \mathrm{d}s = \int_0^1 (\hat{x}_0 + s\Delta\hat{x})(1-\hat{y}_0-s\Delta\hat{y})\mathrm{d}s\nonumber\\
 &= -\frac{\Delta \hat{x} \Delta \hat{y}}{3} + \frac{\Delta \hat{x} (1-\hat{y}_0)}{2} - \frac{\Delta \hat{y} \hat{x}_0}{2} + \hat{x}_0(1-\hat{y}_0)\\
\Phi_{01} &= \int_0^1 \phi_{01}(s) \mathrm{d}s = \int_0^1 (1 - \hat{x}_0 - s\Delta\hat{x})(\hat{y}_0+s\Delta\hat{y})\mathrm{d}s\nonumber\\
 &= -\frac{\Delta \hat{x} \Delta \hat{y}}{3} - \frac{\Delta \hat{x} \hat{y}_0}{2} + \frac{\Delta \hat{y} (1-\hat{x}_0)}{2} + (1-\hat{x}_0)\hat{y}_0\\
\Phi_{11} &= \int_0^1 \phi_{11}(s) \mathrm{d}s = \int_0^1 (\hat{x}_0 + s\Delta\hat{x})(\hat{y}_0+s\Delta\hat{y})\mathrm{d}s\nonumber\\
 &= \frac{\Delta \hat{x} \Delta \hat{y}}{3} + \frac{\Delta \hat{x} \hat{y}_0}{2} + \frac{\Delta \hat{y} x_0}{2} + \hat{x}_0 \hat{y}_0
\end{align}
\end{subequations}

Finally, we may construct the integral of $J_D$ and its derivatives in terms of these $\Phi$ formulae.
\begin{subequations}\label{eqn:JDPhi:all}
\begin{align}
&\int J_D \mathrm{d}r = \|\Delta \p\| \left( X_{i,j} \Phi_{00} + X_{i+1,j} \Phi_{10} + X_{i,j+1} \Phi_{01} + X_{i+1,j+1} \Phi_{11} \right)\\
&\int \frac{\partial J_D}{\partial X_{i,j}} \mathrm{d} r = \|\Delta \p\| \Phi_{00}\\
&\int \frac{\partial J_D}{\partial X_{i+1,j}} \mathrm{d} r = \|\Delta \p\| \Phi_{10}\\
&\int \frac{\partial J_D}{\partial X_{i,j+1}} \mathrm{d} r = \|\Delta \p\| \Phi_{01}\\
&\int \frac{\partial J_D}{\partial X_{i+1,j+1}} \mathrm{d} r = \|\Delta \p\| \Phi_{11}
\end{align}
\end{subequations}


