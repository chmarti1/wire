\section{Introduction}\label{sec:intro}

The works of Clements and Smy on Langmuir probes in dense plasmas \cite{} establish that the saturation current density is a function of its diameter, the fluid velocity, and the local ion density.  For a given wire diameter and fluid velocity, we may infer the ion density from the measured current.

Unfortnately, for a wire injected into a plasma where ion density cannot be assumed constant along its length, it is not possible to distinguish between currents accumulated from base to tip.  Calcote et al. used elaborate methods to insulate and cool wires up to the tip, where the measurement is taken \cite{}, but in the extreme thermal loads observed in atmospheric oxyfuel flames, even these efforts would be doomed.  Good spatial resolution without severe disturbance to the flow demands that probes be tiny in diameter, but the surface-area to cross-sectional area ratio diverges in these geometries.  While this observation makes it all the more impressive that efforts to construct a stationary probe for spatial measurements ever succeeded at all, it compells us to look for alternative methods in atmospheric flames.

If the probe is to be un-cooled, then its time in the flame must be very limited lest it be destroyed.  The imagination may conjure rapid linear injection along the probe's length, but a wire mounted on a disc like was demonstrated by MacLatchey \cite{} is far more mechanically elegant.  Section \ref{sec:speed} addresses the design of a disc to preserve such a probe.

Regardless of how it is injected into the flame, if the probe is a small uninsulated wire, then the measured currents will be an integral of all the currents along its length.  How is it possible to make spatially resolved measurements?  Inspiration may be drawn from tomography and Abel transformation (i.e. ``onion peeling'') methods, which elegantly address the problem of inferring some spatially distributed property from a plurality of integrated line-of-sight measurements.  We develop a framework for deconvolving the measurements numerically in Section \ref{sec:integral}.
