\documentclass{article}

\usepackage{amsmath}
\usepackage{graphicx}

\def\I{\overline{I}}
\def\d{\mathrm{d}}
\def\x{\vec{x}}
\def\vnu{\vec{\nu}_{m,n}}
\def\nuth{\nu_{m,n|\theta}}
\def\ui{\hat{\imath}}
\def\uj{\hat{\jmath}}
\def\real{\mathrm{Re}}
\def\imag{\mathrm{Im}}

\title{A Brief Derivation for Spatial DFT Extraction from Langmuir Probe Currents}
\author{Christopher R. Martin\\Associate Professor of Mechanical Engineering\\Penn State Altoona}
\date{\today}

\begin{document}

\maketitle

\section{Derivation}

The probe wire has radius, $R$, and rotates about a center on the $x$-axis a distance, $d$, from the origin.  At each sample, the probe will have an angle, $\theta$, will enter the domain at radius, $R_0$, and terminates at a radius, $R_1$.  The current measured by the probe, $I$, is an integral of the current per unit wire length, $\I$,
\begin{align}
I = \int_{R_0}^{R_1} \I \d r.
\end{align}
These dimensions are shown in Figure \ref{fig:coords}.  When the tip of the wire lies inside the domain, $R_1 = R$.

\begin{figure}
\centering
\includegraphics[width=.9\linewidth]{figures/coords}
\caption{Dimensions and coordinate system for the domain}\label{fig:coords}
\end{figure}

The wire current density is a function of the bulk velocity and the ion density, so it is expressed as a function of position in the fluid, $\I(\x)$.  Once $\I(\x)$ is known everywhere in the fluid, it is be possible to calculate the ion density distribution.  

Previously, wire current density was calculated as a grid of discrete nodes, and its integral along the wire length was calculated by interpolating along its path.  This approach is intuitive and flexible, and the inversion problem results in a sparse symmetrical matrix, which can speed inversion times.  On the other hand, the path tracing algorithm required iteration along the wire's length, which is slow in high-level languages, the discretization scheme produces artifacts in the images, and there were also unexplained striping artifacts in the wake of strong signals.

In the present work, we adopt a spatial Fourier series, which provides a far more numerically elegant formulation.

\subsection{Fourier series}

The Fourier series for $\I$ is constructed in $x,y$ coordinates
\begin{align}
\I(x,y) &= \sum_{m=-N_x}^{N_x} \sum_{n=-N_y}^{N_y} c_{m,n} \exp\left(2\pi j \vnu \cdot \x \right),
\end{align}
where $\vnu$ is the wavenumber vector,
\begin{align}
\vnu = \frac{m}{N_x L_x} \ui + \frac{n}{N_y L_y} \uj,
\end{align}
and $\x$ is the position in the domain,
\begin{align}
\x = x \ui + y \uj.
\end{align}

This formulation has $(2N_x+1)(2N_y+1)$ complex coefficients, and represents a function with periodicity $L_x$ on $x$ and $L_y$ on $y$.  It is continuous, but cannot resolve features with wavelengths shorter than $L_x / N_x$ and $L_y / N_y$.  In this way, the results are effectively spatially filtered by the highest wavenumber in the expansion.

Because the coefficients are complex-valued, each of the $(2N_x+1)(2N_y+1)$ coefficients represents two unknowns.  However, because the function, $\I$, is strictly real-valued, the coefficients must occur in complex conjugate pairs.  As we will examine in more depth later, the number of unknonws only scales like $4N^2$ (when $N = N_x = N_y$)

\subsection{Evaluating the integral}

The integral of $\I$ requires an $x,y$ parametric formulation of the wire's path.  At an angle, $\theta$, the wire is oriented along a unit vector,
\begin{align}
\hat{e}_\theta = \cos(\theta) \ui + \sin(\theta) \uj.
\end{align}


If $r$ is the distance along the wire from the disc center, the corresponding location in $(x,y)$ is
\begin{align}
\x &= r \hat{e}_\theta - d \ui
\end{align}
The wavenumber along the wire (in the $\hat{e}_\theta$ direction) can be written as a scalar,
\begin{align}
\nuth &= \vnu \cdot \hat{e}_\theta\nonumber\\
 &= \frac{m}{L_x} \cos(\theta) + \frac{n}{L_y} \sin(\theta).
\end{align}

The radius at which the wire crosses into the domain, $R_0$, the radius at which the wire either terminates or leaves the domain, $R_1$, and the wire length in the domain, $\Delta R$, are
\begin{align}
R_0 &= \frac{d}{c_\theta}\\
R_1 &= \min\left(\ R,\ \frac{d + L_x}{\cos(\theta)},\ \left|\frac{L_y}{2\sin(\theta)}\right|\ \right)\\
\Delta R &= R_1 - R_0.
\end{align}
The integral is a sum of surface currents along the wire's length at moment in time, so the disc position parameters, $d$ and $\theta$, are constant.  For compactness of notation, it will become convenient to express the trigonometric functions on $\theta$ as constants, $c_\theta$ and $s_\theta$.

A single term of the Fourier series appears
\begin{align*}
&c_{m,n} \exp\left(2\pi j \vnu \cdot \x \right)\\
&\hspace{2em}= c_{m,n} \exp\left(2\pi j \vnu \cdot (\hat{e}_\theta r - d \ui) \right)\\
&\hspace{2em}= c_{m,n} \exp\left(-2\pi j \frac{md}{L_x} \right) \exp\left(2\pi j \nuth r \right)
\end{align*}


For an integral over $r$, all but the last portion of the term above is constant, so it is convenient to define a new parameter, $\gamma$, which represents the value of this term integrated over $r$:
\begin{align}
\gamma_{m,n}(d,\theta) &= \int_{R_0}^{R_1} \exp\left(2\pi j \nuth r \right) \d r \nonumber\\
 &= \left\{\begin{array}{c|l}
 \Delta R & \nuth = 0\\
 \left.\frac{1}{2\pi j \nuth}\exp\left(2\pi j \nuth r \right)\right|^{R_1}_{R_0} & \mathrm{otherwise}
\end{array}
\right.
\end{align}
The wavenumber, $\nuth$, can be zero when $m=n=0$ or when $\x$ is exactly normal to the wavenumber vector.

When we express the complex coefficient, $c_{m,n}$ in its real and imaginary parts,
\begin{align}
c_{m,n} = a_{m,n} + j b_{m,n},
\end{align}
this yields an expression for the total wire current
\begin{align}
I(d,\theta) &= \sum_{m=-N_x}^{N_x} \exp\left(-2\pi j \frac{md}{N_x L_x} \right) \sum_{n=-N_y}^{N_y} c_{m,n} \gamma_{m,n}(d,\theta)
\end{align}

\subsection{Offset current}

Every experiment begins by zeroing the current signal to the nearest practical precision, but no real signal will ever be perfectly zero to numerical precision.  In most applications, this is not especially problematic, but the derivation above makes no allowance for any component of current signal that appears on portions of the wire where $x$ is negative.  For this reason, it is prudent to add a global offset parameter, which is intended to be a small current that is present everywhere.  

\begin{align}
I(d,\theta) &= I_0 + \sum_{m=-N_x}^{N_x} \exp\left(-2\pi j \frac{md}{L_x} \right) \sum_{n=-N_y}^{N_y} c_{m,n} \gamma_{m,n}(d,\theta)
\end{align}

\subsection{Alternate realization}

Because the $c_{m,n}$ coefficients must occur in complex conjugate pairs, half of the $2(2N_x+1)(2N_y+1)$ unknown values represent redundant information.  If we consider wavenumber vectors that are negatives of each other, note that corresponding values of $\gamma$ will be complex conjugates,
\begin{align}
\gamma_{-m,-n} = \gamma^*_{m,n}.
\end{align}
Figure \ref{fig:wavenumbers} shows these complimentary $\vnu$ vectors in wavenumber space that correspond to coefficients with complex conjugates.  Note that

It is possible, therefore, to write
\begin{align}
I(d,\theta) &= I_0 + c_{0,0} \nonumber\\
& \hspace{-1em} + \sum_{n=1}^{N_y} c_{0,n} \gamma_{0,n} + c_{0,n}^* \gamma^*_{0,n} \nonumber\\
& \hspace{-1em} + \sum_{m=1}^{N_x} \sum_{n=-N_y}^{N_y} c_{m,n} \exp\left(-2\pi j \frac{m d}{L_x}\right) \gamma_{m,n} +  c^*_{m,n} \exp\left(2\pi j \frac{m d}{L_x}\right) \gamma^*_{m,n}
\end{align}

\begin{figure}
\centering
\includegraphics[width=0.9\linewidth]{figures/wavenumbers}
\caption{The wave number space with the included terms marked in green and their negative mirrors marked in red.}\label{fig:wavenumbers}
\end{figure}

\section{Inversion}

Fourier transforms classically take advantage of the orthogonality of the sinusoidal basis functions, so the original signal only needs to be integrated against each basis function over the entire domain to calculate the magnitude and phase of that component.  Because no single wire position spans the entire domain, this approach is not tenable.

Instead, we benefit from the fact that the unknown parameters are linear coefficients.  If they were to be serialized in a vector, $\vec{x}$, the problem lends itself to a simple least-squared approach.  Because they are complex each instance of $c_{m,n}$ actually represents two unknown coefficients,
\begin{align}
c_{m,n} = a_{m,n} + j b_{m,n}.
\end{align}
The scheme used to organize the coefficients in $\vec{x}$ is not especially important, but it might be something along the lines of 
\begin{align}
\vec{x} = \{a_{0,0},\ I_0,\ a_{1,0},\ b_{1,0},\ a_{2,0},\ b_{2,0},\ \ldots \}^T.
\end{align}
Note that $b_{0,0}$ has been replaced by the offset current, $I_0$.  When $m=n=0$, the expression is purely real, so $b_{0,0}$ is ignored by this formulation - the solution is independent of its selection.  So, with the addition of $I_0$, there are $2(N_x+1)(N_y+1)$ coefficients to be calculated.

When the terms of (\ref{eqn:I}) are embedded into a second vector, $\vec{\Lambda}$, the equation may be rewritten
\begin{align}
I(d,\theta) = \vec{\Lambda}(d,\theta) \cdot \vec{x}.
\end{align}
Here, $\vec{\Lambda}$, is a vector that models the contribution of each coefficient to the current of a wire in location $(d,\theta)$.

For a given wire position, $(d_i, \theta_i)$, there will be a measured current, $I_i$.  For a given coefficient set, there will be an error,
\begin{align}
e_i = I_i - \vec{\Lambda}(d_i, \theta_i) \cdot \vec{x}.
\end{align}
In a least squares approach, we differentiate the sum of the squares of errors for each of the data points.  For compactness of notation, it will be convenient to abbreviate $\vec{\Lambda}_i = \vec{\Lambda}(d_i, \theta_i)$ moving forward.

\begin{align}
\vec{0} = \nabla \sum e_i {^2} = \sum 2(I_i - \vec{\Lambda}_i \cdot \vec{x})(- \vec{\Lambda}_i)
\end{align}
Solving for $\vec{x}$,
\begin{align}
\vec{x} = \left(\sum \vec{\Lambda}_i \vec{\Lambda}_i{^T}\right)^{-1} \sum I_i \vec{\Lambda}_i
\end{align}

The multiplication of $\Lambda$ with its transpose forms a symmetrical matrix.  The summation accumulate over the body of data collected to form a matrix that must be inverted.

\end{document}
