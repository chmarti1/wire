\documentclass{article}

\usepackage{amsmath}
\usepackage{graphicx}

\def\I{\overline{I}}
\def\d{\mathrm{d}}
\def\x{\vec{x}}
\def\eth{\hat{e}_\theta}
\def\vnu{\vec{\nu}_{m,n}}
\def\nuth{\nu_{m,n|\theta}}
\def\ui{\hat{\imath}}
\def\uj{\hat{\jmath}}
\def\real{\mathrm{Re}}
\def\imag{\mathrm{Im}}

\title{A Brief Derivation for Spatial DFT Extraction from Langmuir Probe Currents}
\author{Christopher R. Martin\\Associate Professor of Mechanical Engineering\\Penn State Altoona}
\date{\today}

\begin{document}

\maketitle

\section{Derivation}

The probe wire has radius, $R$, and rotates about a center on the $x$-axis a distance, $d$, from the origin.  At each sample, the probe will have an angle, $\theta$, will enter the domain at radius, $R_0$, and terminates at a radius, $R_1$.  The current measured by the probe, $I$, is an integral of the current per unit wire length, $\I$,
\begin{align}
I = \int_{R_0}^{R_1} \I \d r.
\end{align}
These dimensions are shown in Figure \ref{fig:coords}.  When the tip of the wire lies inside the domain, $R_1 = R$.

\begin{figure}
\centering
\includegraphics[width=.9\linewidth]{figures/coords}
\caption{Dimensions and coordinate system for the domain}\label{fig:coords}
\end{figure}

The wire current density is a function of the bulk velocity and the ion density, so it is expressed as a function of position in the fluid, $\I(\x)$.  Once $\I(\x)$ is known everywhere in the fluid, it is be possible to calculate the ion density distribution.  

Previously, wire current density was calculated as a grid of discrete nodes, and its integral along the wire length was calculated by interpolating along its path.  This approach is intuitive and flexible, and the inversion problem results in a sparse symmetrical matrix, which can speed inversion times.  On the other hand, the path tracing algorithm required iteration along the wire's length, which is slow in high-level languages, the discretization scheme produces artifacts in the images, and there were also unexplained striping artifacts in the wake of strong signals.

In the present work, we adopt a spatial Fourier series, which provides a far more numerically elegant formulation.

\subsection{Fourier series}

The Fourier series for $\I$ is constructed in $x,y$ coordinates
\begin{align}
\I(x,y) &= \sum_{m=-N_x}^{N_x} \sum_{n=-N_y}^{N_y} c_{m,n} \exp\left(2\pi j \vnu \cdot \x \right),\label{eqn:Ibar}
\end{align}
where $\vnu$ is the wavenumber vector,
\begin{align}
\vnu &= \nu_x \ui + \nu_y \ui\nonumber\\
 &=\frac{m}{L_x} \ui + \frac{n}{L_y} \uj,
\end{align}
and $\x$ is the position in the domain,
\begin{align}
\x = x \ui + y \uj.
\end{align}

This formulation has $(2N_x+1)(2N_y+1)$ complex coefficients, and represents a function with periodicity $L_x$ on $x$ and $L_y$ on $y$.  It is continuous, but cannot resolve features with wavelengths shorter than $L_x / N_x$ and $L_y / N_y$.  In this way, the results are effectively spatially filtered by the highest wavenumber in the expansion.

Because the coefficients are complex-valued, each of the $(2N_x+1)(2N_y+1)$ coefficients represents two unknowns.  However, because the function, $\I$, is strictly real-valued, the coefficients must occur in complex conjugate pairs.  As we will examine in more depth later, the number of unknonws only scales like $4N^2$ (when $N = N_x = N_y$)

\subsection{Evaluating the integral}

The integral of $\I$ requires an $x,y$ parametric formulation of the wire's path.  At an angle, $\theta$, the wire is oriented along a unit vector,
\begin{align}
\hat{e}_\theta = \cos(\theta) \ui + \sin(\theta) \uj.
\end{align}

If $r$ is the distance along the wire from the disc center at $x=-d$, the corresponding location in $(x,y)$ is
\begin{align}
\x &= r \eth - d \ui
\end{align}
The dot product of $\x$ and $\vnu$ yields two terms along the wire's path; one with the wavenumber along the wire (in the $\eth$ direction) and one with the wavenumber in the horizontal axis (in the $\ui$ direction),
\begin{align}
\vnu \cdot \x &= r (\vnu \cdot \hat{e}_\theta) - d (\vnu \cdot \ui)\nonumber\\
 &= \nu_{m,n|\theta} r - \nu_{m,n|x} d\\
\nu_{m,n|\theta} &= \frac{m}{L_x} \cos(\theta) + \frac{n}{L_y} \sin(\theta)\\
\nu_{m,n|x} &= \frac{m}{L_x}.
\end{align}
In the domain, the term still has its two-dimensional wave properties, but if we were to travel only along a wire at angle, $\theta$, the $(m,n)$ term of the expansion would appear to have this wave number.  Note that it can be zero, even when both $x-$ and $y-$components of $\vnu$ are non-zero; when $\vnu$ is normal to $\eth$.

Once substituted into the Fourier expansion, a single term becomes
\begin{align*}
c_{m,n} \exp\left(2\pi j \vnu \cdot \x \right)\hspace{-8em}&\\
&= c_{m,n} \exp\Bigl[2\pi j \left(\nuth\,r - \nu_{m,n|x} d\right)\Bigr].
\end{align*}
Here, the disc center location, $d$, appears intuitively as a phase shift in the horizontal direction only.  Recall that $\nu_{m,n|x}$ is simply $m / L_x$.

In order to integrate along this function, we need to establish the radius at which the wire crosses into the domain, $R_0$, the radius at which the wire either terminates or leaves the domain, $R_1$:
\begin{align}
R_0 &= \frac{d}{\cos(\theta)}\\
R_1 &= \min\left(\ R,\ \frac{d + L_x}{\cos(\theta)},\ \left|\frac{L_y}{2\sin(\theta)}\right|\ \right)
\end{align}
The integral is a sum of surface currents along the wire's length at moment in time, so the disc position parameters, $d$ and $\theta$, are constant.

It is convenient to define a new parameter, $\lambda_{m,n}$, which represents the value of the exponential integrated over $r$:
\begin{align}
\lambda_{m,n}(d,\theta) &= \int_{R_0}^{R_1} \exp\Bigl[2\pi j \left(\nuth\,r - \nu_{m,n|x}\,d\right)\Bigr] \d r \nonumber\\
 &= \left\{\begin{array}{c@{\ :\ }l}
 \displaystyle (R_1 - R_0) \exp\Bigl[-2\pi j \nu_{m,n|x}\,d\Bigr] & \nuth = 0  \vspace{1.0em} \\
\displaystyle \frac{1}{2\pi j \nuth}\exp\Bigl[2\pi j \left(\nuth\,r - \nu_{m,n|x}\,d\right)\Bigr]^{R_1}_{R_0} & \mathrm{otherwise}
\end{array}
\right.
\end{align}
It is worth repeating that the wavenumber, $\nuth$, can be zero when $m=n=0$ or when $\eth$ is exactly normal to the wavenumber vector.

This yields an expression for the total wire current
\begin{align}
I(d,\theta) &= \sum_{m=-N_x}^{N_x} \sum_{n=-N_y}^{N_y} c_{m,n} \lambda_{m,n}(d,\theta)
\end{align}

\subsection{Offset current}

Every experiment begins by zeroing the current signal to the nearest practical precision, but no real signal will ever be perfectly zero to numerical precision.  Normally, this would not be especially problematic, but the derivation above provides no representation of current sources outside of the domain.  To prevent the kinds of bizarre numerical phenomena that can occur when models awkwardly attempt to match behaviors beyond their capability, it is prudent to add a global offset parameter, $I_0$,  
\begin{align}
I(d,\theta) &= I_0 + \sum_{m=-N_x}^{N_x} \sum_{n=-N_y}^{N_y} c_{m,n} \lambda_{m,n}(d,\theta)\label{eqn:I}.
\end{align}

\section{Inversion}

Fourier transforms classically take advantage of the orthogonality of the sinusoidal basis functions, so the original signal only needs to be integrated against each basis function over the entire domain to calculate the magnitude and phase of that component.  Because no single wire position spans the entire domain, this approach is not tenable.  Instead, a vector of coefficients must be solved by inverting a matrix.

The exact scheme used to organize the coefficients into a vector, $\vec{c}$, is not especially important, but it might be something along the lines of 
\begin{align}
\vec{c} = [c_{-N_x,-N_y},\ c_{-N_x+1, -N_y},\ \ldots,\ c_{-1,0},\ c_{0,0},\ c_{1,0},\ \ldots,\ c_{N_x,N_y},\ I_0]^T.
\end{align}
In this arrangement, it is equally valid to refer to a coefficient by its wavenumber indices, $c_{m,n}$, or by its place in the vector $c_k$.  When the indexing scheme above is used,
\begin{align}
k &= (m + N_x) + (n+N_y) (2N_x+1)\\
m + N_x &= k \mod{(2N_x+1)}\\
n + N_y &= \mathrm{floor}(k / (2N_x+1))
\end{align}

When the $\lambda$ values from (\ref{eqn:I}) are embedded into a second vector, $\vec{\Lambda}$, the equation may be rewritten
\begin{align}
I(d,\theta) = \vec{\Lambda}(d,\theta) \cdot \vec{c}.
\end{align}
Here, $\vec{\Lambda}$, is a vector that models the contribution of each coefficient to the current of a wire in location $(d,\theta)$.

For a given wire position, $(d_i, \theta_i)$, there will be a measured current, $I_i$.  For a given coefficient set, there will be an error,
\begin{align}
e_i = I_i - \vec{\Lambda}(d_i, \theta_i) \cdot \vec{c}.
\end{align}
For compactness of notation, it will be convenient to abbreviate $\vec{\Lambda}_i = \vec{\Lambda}(d_i, \theta_i)$ moving forward.  

Though other methods exist, least squares regression is the most common method for extracting optimal coefficients.  Using this approach, the sum of the squares of the errors, $\sum e_i{^2}$, is differentiated with respect to each of the coefficients.  When error is minimal, that differential is zero.  So,
\begin{align}
0 = \frac{\partial}{\partial c_k} \sum e_i {^2} = \sum 2\left(I_i - \vec{\Lambda}_i \cdot \vec{c}\right)\left(- \lambda_{i,k}\right).
\end{align}

Note that $\lambda_{i,k}$ is the element of $\vec{\Lambda}_i$ that corresponded to the coefficient being differentiated.  When this is repeated for each of the coefficients, and the resulting system of equations is written as a matrix problem,
\begin{align}
\vec{0} = -\sum I_i \vec{\Lambda}_i + \left(\sum \vec{\Lambda}_i \vec{\Lambda}_i{^T} \right) \vec{c}
\end{align}

The multiplication, $\vec{\Lambda} \vec{\Lambda}^T$, forms a symmetrical matrix, and the sum of many symmetrical matrices is also a symmetrical matrix.  However, because the problem is complex-valued, it is preferable to obtain a Hermitian matrix if possible.  Fortunately, it is equally valid to differentiate with respect to the complex conjugate, $\overline{c_k}$, instead of $c_k$.  This yields the problem
\begin{align}
\vec{0} = -\sum I_i \overline{\vec{\Lambda}_i}  + \left(\sum \overline{\vec{\Lambda}_i} \vec{\Lambda}_i{^T} \right) \vec{c}.
\end{align}

The sum of both the constant vector (left-hand term) and the solution matrix (right-hand term) may be accumulated over hundreds of thousands of individual measurements, so accumulating the matrix is typically a more computationally intensive task than solving the linear algebra problem.



\end{document}
